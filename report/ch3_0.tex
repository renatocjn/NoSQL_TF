\chapter{Exemplo de aplicação}
Neste capítulo, descreveremos a construção de um simples aplicativo web para gerenciamento de tarefas. Mostraremos como acessar a API do Amazon SimpleDB usando a linguagem python e os resultados obtidos.\

\section{Python API}

\subsection{The Boto API}
Boto é um projeto que proporciona API's simples de manuseio dos web services da amazon para a linguagem python

\subsection{Pricipais Classes}

\subsubsection{boto.sdb.connection}
Proporciona a abstração da conexão com o amazon simple db com a sintaxe:\\\\
\texttt{sdb.connection.SDBConnection(aws$\_$access$\_$key$\_$id=None, aws$\_$secret$\_$access$\_$key=None, is$\_$secure=True, port=None, proxy=None, proxy$\_$ port=None, proxy$\_$ user=None, proxy\_ pass=None, debug=0, https$\_$ connection$\_$ factory=None, region=None, path='/', converter=None, security$\_$ token=None, validate$\_$ certs=True)}
\\\\
Mas \'e normalmente usado na forma:\\
\texttt{sdb.connection.SDBConnection('your$\_$access$\_$key', 'your$\_$private$\_$access$\_$key', region='sua regiao de preferencia')}
\\
A regi\~ao \'e uma das listadas na documenta\cc\~ao
\\\\
Dada uma conex\~ao voc\~e pode executar as opera\cc\~oes descritas abaixo:
\begin{itemize}
 \item batch$\_$delete$\_$attributes(domain$\_$or$\_$name\footnotemark[1], items\footnotemark[2])
 \item batch$\_$put$\_$attributes(domain$\_$or$\_$name, items, replace=True)
 \item create$\_$domain(domain$\_$name)
 \item delete$\_$attributes(domain$\_$or$\_$name, item$\_$name, attr$\_$names=None, expected$\_$value=None)
 \item delete$\_$domain(domain$\_$or$\_$name)
 \item domain$\_$metadata(domain$\_$or$\_$name)
 \item get$\_$all$\_$domains(max$\_$domains=None, next$\_$token=None)
 \item get$\_$attributes(domain$\_$or$\_$name, item$\_$name, attribute$\_$names=None, consistent$\_$read=False, item=None)
 \item get$\_$domain(domain$\_$name, validate=True)
 \item put$\_$attributes(domain$\_$or$\_$name, item$\_$name, attributes\footnotemark[2], replace=True, expected$\_$value=None)
 \item select(domain$\_$or$\_$name, query='', next$\_$token=None, consistent$\_$read=False)
 
 \footnotetext[1]{'domain$\_$or$\_$name' deve ser uma string com o nome do dominio ou o proprio objeto de dominio, como descrito na proxima se\cc\~ao}
 \footnotetext[2]{'items' e 'attributes' deve ser um objeto dict-like}
\end{itemize}

\subsubsection{boto.sdb.domain}
Essa classe forne\cce a abstração dos dominios registrados na sua conta aws e permite a execu\cc\~ao de das mesmas instrucoes de manipula\cc\~ao de dados presentes na classe boto.sdb.connection, a cria\cc\~ao desse objeto \'e possivel com a chamada:\\
\texttt{myconnection.get$\_$domain('mydomain')}
\\\\
