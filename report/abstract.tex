\begin{abstract}
  A criação e a manutenção de uma infraestrutura de banco de dados costuma ser um dos principais gastos(em termos de tempo e finanças) de sistemas de software comerciais modernos. Os gastos incluem a aquisição da plataforma de hardware necessária para suportar o sistema de gerenciamento de banco de dados(\textit{SGBD}), conexões de rede e --- dependendo do porte do sistema --- contratação de um administrador de banco de dados(\textit{DBA}). Tais gastos são rotineiramente um fator limitante para a criação de sistemas comerciais por \textit{startups} e empresas de pequeno porte que não têm condições de arcar com esses custos.\\\\
  
  Com a popularização da computação em nuvem e a oferta de serviços segundo o modelo PaaS(\textit{Platform as a Service}), surgiram alternativas ao modelo de banco de dados \textit{in house} mencionado acima que provêem infraestruturas de bancos de dados como um serviço, geralmente cobrando de acordo com a carga de trabalho utilizada(armazenamento, transferência de dados, processamento, etc).\\\\
  
  Este relatório técnico tem como objetivo prover uma visão geral sobre as características, o conceito e o uso de um serviço específico desse gênero, o banco de dados NoSQL em nuvem \textbf{Amazon SimpleDB}, bem como esclarecer o contexto em que a ferramenta se apresenta.
\end{abstract}